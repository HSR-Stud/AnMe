% !TeX spellcheck = de_CH_frami
\section{Stabilität von MOS Operationsverstärker (Kap. 13)}
Loop-Gain $T(s)$: $T(s) = A(s)\cdot F(s)$\\
\begin{tabular}{|l|l|}
	\hline
	Phasenmarge bei& Verhalten des Verstärkers\\
	$f_{krit}$ ($a_L = 1$)& (System mit zwei weit auseinanderliegenden Polen)\\ \hline
	$\varphi_M \leq \SI{0}{\degree}$& Gegenkoppelter Verstärker schwingt selbständig\\ \hline
	$\varphi_M > \SI{0}{\degree}$& Gedämpftes Überschwingen der Sprungantwort\\ \hline
	$\varphi_M = \SI{65}{\degree}$& Peaking verschwindet. Einziger Überschwinger mit \SI{4.7}{\percent} Sprunghöhe\\ \hline
	$\varphi_M \geq \SI{75}{\degree}$& Kein Überschwingen\\ \hline
\end{tabular}\\[2ex]
\begin{tabular}{ll}
	Stabilitätskriterien&$\phi = \SI{180}{\degree} => |A(s)\cdot F(s)| < 1$\\
	&$|A(s)\cdot F(s)| = 1 => \SI{180}{\degree}-\Phi > 0; \varphi_M > \SI{0}{\degree}$\\
	Phasenmarge&$\varphi_M = \SI{180}{\degree}-\Phi = \SI{90}{\degree}-arctan(\frac{GBP}{f_{P2}})$\\
	Designregel&Wähle 2. Pol ($f_{nd}$) bei ca. $3\cdot GBP$\\
	&Dies ergibt eine Phasenmarge von \SI{72}{\degree} und somit kaum Überschwingen
\end{tabular}\\[2ex]
\begin{minipage}[c]{0.45\textwidth}
	Jeder Knoten $N$ bildet einen Pol bei der Frequenz $f_N$, der sich wie folgt berechnet:
	$f_N=\frac{1}{2\pi\cdot R_N C_N}$\\[2ex]
	\textbf{Grobe Analyse:} Die Knoten mit hohen RC-Produkten suchen. Dort entstehen Systempole, welche einen Abfall von \SI{20}{\decibel/Dekade} im Frequenzgang einleiten.
	\subsection{Widerstände}
	\textbf{Knotenimpedanz praktisch unendlich:}\\
	Gate: $r_{iG} -> \infty$\\
	\textbf{Knotenimpedanz sehr hoch:}\\
	Drain des Transistors wenn als Stromquelle beschaltet: $r_{ds}= \frac{1}{g_0}$\\
	\textbf{Knotenimpedanz tief:}\\
	Drain des Transistor in Diodenschaltung, Source des Transistors in Stromquellenschaltung: $\frac{1}{g_m}$
	
	\subsection{Kapazitäten}
	\textbf{Knotenkapazität gross:}\\
	C als passive Schaltungskomponente\\
	\textbf{Knotenkapazität mittel:}\\
	parasitäre Kapazität verstärkt durch Miller-Effekt. Häufig $C_{GD}$ eines verstärkenden Transistors.\\
	\textbf{Knotenkapazität klein:}\\
	Knoten mit parasitären Kapazitäten. Von diesen Knoten ist in der Regel der Gate-Knoten mit der höchsten Kapazität belastet.
\end{minipage}
\begin{minipage}[c]{0.5\textwidth}
	\subsection{Miller-Effekt}
	Die Miller-Kapazität $C_m$ zwischen Ein- und Ausgang eines Verstärkers mit Verstärkung A liegt, erscheint
	\begin{itemize}
		\item multipliziert mit (1-A) parallel zum Eingang ($C_{mi}$)
		\item multipliziert mit (1-1/A) parallel zum Ausgang ($C_{mo}$)
		\item $C_m$ wird aus dem Schema entfernt und durch $C_{mi}$ und $C_{mo}$ ersetzt.
	\end{itemize}
	\subsection{Darstellung}
	1. DC Verstärkung berechnen (aus Kleinsignalersatzschaltung für Niederfrequenz)\\
	2. Die für die Übertragungsfunktion relevanten Pole finden. (An welchem Knoten befindet sich ein hohes RC Produkt?)\\
	3. Pole im Bode-Diagramm einzeichnen
\end{minipage}\\
\subsection{Typische Kapazitäten}
\begin{tabular}{|l|l|l|l|l|}
	\hline
	&$C_{GS}$&$C_{GD}$&$C_{SB}$&$C_{DB}$\\ \hline
	Gesättigt&$C_{GS0} + \frac{2}{3} C_{oxt}$&$C_{GD0}$&$C_{jSBt}+\frac{2}{3}C_{BCt}$&$C_{jDBt}$\\
	Typ. Wert&$\SI{33}{\femto\farad}$&$\SI{1.2}{\femto\farad}$&$\SI{10}{\femto\farad}$&$\SI{7}{\femto\farad}$\\ \hline
	Ungesättigt&$C_{GS0}+\frac{1}{2}C_{oxt}$&$C_{GD0}+\frac{1}{2}C_{oxt}$&$C_{jSBt}+\frac{1}{2}C_{BCt}$&$C_{jDBt}+\frac{1}{2}C_{BCT}$\\
	Typ. Wert&$\SI{26}{\femto\farad}$&$\SI{26}{\femto\farad}$&$\SI{10}{\femto\farad}$&$\SI{10}{\femto\farad}$\\ \hline
\end{tabular}\\
$C_{oxt}=C_{ox}\cdot W \cdot L_{eff}$ \hspace{0.5mm} $C_{BCt}=C_{jBC}\cdot W \cdot L_{eff}$\\
$C_{jSBt} = C_{jSB} \cdot A_S + C_{jswSB} \cdot P_S$\\
$C_{jDBt} = C_{jDB}\cdot A_D + C_{jswDB}\cdot P_D$\\

Wenn $V_{SB}=0$, dann $C_{SB}$ ignorieren und $C_{DB}=C_{DS}$.